En el siguiente trabajo pr\'actico se realizar\'a un an\'alisis sobre el tr\'afico de paquetes ARP de una red local WI-FI con el fin de estudiar dos fuentes de informaci\'on. Para recolectar los datos implementamos un \"sniffer\", es decir un programa que toma todos los paquetes de la red. La implementaci\'on del sniffer la realizamos en lenguaje Pyhton, porque hay una biblioteca llamada Scapy \footnote{https://github.com/secdev/scapy/} que posee varias funcionalidades que son de mucha utilidad para manejar paquetes. 

\subsection{ARP: Adress Resolution Protocol}

Es el protocolo encargado de \"mapear\" direcciones del nivel 3 al nivel 2. Los paquetes ARP poseen varios campos de informaci\'on pero en este trabajo s\'olo nos van a interesar la direcci\'on MAC e IP destino y fuente y el tipo de mensaje, es decir si es \"who-has\" o \"is-at\".
Para nuestros experimentos s\'olo nos vamos a enfocar en los paquetes de tipo who-has dado se env\'ian de manera \emph{broadcast} y los is-at son \emph{Unicast}. Cuando un equipo quiere comunicarse con otro env\'ian un paquete ARP de tipo who-has y el equipo buscado responde con un paquete ARP de tipo is-at. Estos env\'ios isrven para que cada equipo arme su propia tabla ARP en la cual asocia direcciones IP con direcciones MAC.

\subsection{Teor\'ia de la informaci\'on}

Para este trabajo vamos a usar la teor\'ia estad\'istica de la informaci\'on (o teor\'ia de Shannon). Nos interesan las siguientes definiciones:

\begin{itemize}

\item Informaci\'on de un suceso: Dado un suceso $i$ de una fuente cuya probabilidad es estrictamente mayor que cero se define su informaci\'on como $c_i = -log(p_i)$ con $p_i$ la probabilidad de que ocurra el suceso $i$. Se puede observar que cuanto menos probable sea un suceso m\'as informaci\'on provee y cuanto m\'as probable sea menos informaci\'on aporta.

\item Entrop\'ia de una fuente: Dada una fuente $S$ se define la entrop\'ia como $H(S) = \sum{i=1}{i=\#(S)} p_i*c_i$ con $c_i$ la informaci\'on del suceso $i$ y $p_i$ su probabilidad. Esta m\'etrica es un promedio ponderado de la informaci\'on que brinda cada suceso.  

\end{itemize}

