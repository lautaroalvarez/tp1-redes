\section{Introducción}

\par En el siguiente trabajo práctico vamos a análizar 3 redes locales con el fin de caracterizarlas en base a los nodos que participan enviando o recibiendo paquetes, la cantidad de paquetes y el rol que cumplen (o creemos que cumplen) dentro de la red. Para esto, utilizaremos un \textit{sniffer} que escucha la red y almacena los datos recolectados.
\par Antes que nada, daremos unos conceptos importantes sobre el tema para poder explicar mejor el funcionamiento y alcance de esta herramienta.

\subsection{ARP: Adress Resolution Protocol}

\par Es el protocolo encargado de mapear direcciones del nivel 3 al nivel 2. Los paquetes ARP poseen varios campos de información pero en este trabajo solo nos van a interesar la dirección MAC e IP destino y fuente y el tipo de mensaje, es decir si es who-has o is-at.
\par Cuando un equipo quiere comunicarse con otro envía un paquete ARP de tipo who-has (broadcast) y el equipo buscado responde con un paquete ARP de tipo is-at (unicast). Estos envíos sirven para que cada equipo arme su propia tabla ARP en la cual asocia direcciones IP con direcciones MAC.

\subsection{Modo promiscuo}

\par Se trata de una configuración de la placa de red, en la cual transmite a sus niveles superiores todos los paquetes que escucha, y no solo los que van dirigidos a nuestra pc (como pasa normalmente). La herramienta sniffer utiliza precisamente este modo para tomar todos los paquetes que logre leer.

\subsection{Teoría de la información}

\par Vamos a usar la teoría estadística de la información (o teoría de Shannon). Sin entrar tan en detalle, nos interesarán las siguientes definiciones:

\begin{itemize}
  \item \textbf{Información de un suceso:} Dado un suceso $i$ de una fuente cuya probabilidad es estrictamente mayor que cero se define su información como $c_i $= $-log(p_i)$ con $p_i$ la probabilidad de que ocurra el suceso $i$. Se puede observar que cuanto menos probable sea un suceso más información provee y cuanto más probable sea menos información aporta.
  \item \textbf{Entropía de una fuente:} Dada una fuente $S$ se define la entropía como $H(S)$ = $\sum_{i=1}^{\#(S)} p_i*c_i$ con $c_i$ la información del suceso $i$ y $p_i$ su probabilidad. Esta m\'etrica es un promedio ponderado de la información que brinda cada suceso.
\end{itemize}

\subsection{Herramienta sniffer}
\par Ahora sí, podemos pasar a definir la herramienta que desarrollamos y una breve introducción a su funcionamiento. La herramienta utiliza una librería llamada scapy\footnote{SCAPY: https://github.com/secdev/scapy/}, que es la encargada de pasar al sistema a modo promiscuo y obtener todos los paquetes que pueda identificar. De esta manera, luego de obtener estos paquetes, se calculan la información y entropía de distintas fuentes de información (que luego mencionaremos) y al mismo tiempo se almacena la información que consideramos relevante en un archivo de salida (para poder ser utilizado luego).

\subsection{Fuentes de información analizadas}
\par Para analizar los datos de paquetes obtenidos por la herramienta definimos dos fuentes de información:
\begin{itemize}
  \item \textbf{S:} El conjunto de símbolos definidos es $\{S_{broadcast},S_{Unicast}\}$. El símbolo $S_{broadcast}$ corresponde a los paquetes capturados cuya dirección de destino es ff:ff:ff:ff:ff:ff, mientras que el resto de los paquetes van a ser considerados $S_{unicast}$.
  \item \textbf{S1:} El conjunto de símbolos definidos son todas las direcciones IPv4 que tuvieron todos los dispositivos mientras estaban conectados a la red. En esta fuente sólo consideramos los paquetes de tipo \textbf{who-has}.
\end{itemize}
