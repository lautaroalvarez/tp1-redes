\section{Conclusiones generales}

\par En este informe realizamos un recorrido por cuatro tipos de redes locales con características y dimensiones distintas. La idea inicial era que las redes fueran bastante disímiles para poder contrastarlas e identificar rasgos que nos resulten llamativos. Nos basamos principalmente en el tamaño de la red y no en la tecnología de los equipos que se encontraban en ellas. Sin embargo, al trabajar sobre redes grandes notamos algunos factores que nos daban información sobre la organización de dichas redes y, luego de analizar esto, dimos una opinión sobre los posibles equipos que participan en ella.

\par Luego de observar el comportamiento de las redes con mayor cantidad de nodos, notamos que los mensajes Broadcast superan ampliamente a los mensajes Unicast. Esto se lo debemos, como es de esperarse, a los mensajes de control que se necesitan para mantener la red funcionando correctamente y pudimos notar el impacto directo que tiene en la información de dicho símbolo (\textit{Broadcast}). En contrapartida, la información de los mensajes Unicast es elevada, y todo esto hace que la entropía diste mucho mas de su valor máximo (\textit{0.5}).

\par Para el caso de la fuente S1, notamos que cuanto mas grande es la red, mas notamos la distinción de un nodo router o access point (que en general es planteado como el gran candidato a ser un nodo distinguido). De la misma manera, se observaron algunos otros nodos que se acercaban (que luego intentamos identificar como "organizadores" de la red) y muchos nodos que se comportan de manera similar (los host consumidores de la red).

\par Sobre la misma idea de distinguir nodos, este trabajo nos deja la idea de que la comparación de la información de los distintos símbolos (hosts) con la entropía de la fuente S1 puede brindarnos información relevante sobre los nodos distinguidos de la red. Y al seguir por ese camino, cuando ordenamos los hosts por su información notamos que varios, a pesar de posicionarse por encima de la entropía, también brindaban información sobre la organización de la red.
